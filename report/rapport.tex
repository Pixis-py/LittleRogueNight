\documentclass[10pt]{article}

\usepackage[T1]{fontenc}
\usepackage[utf8]{inputenc}
\usepackage{color}
\usepackage{graphicx}
\usepackage[french]{babel}
\usepackage{hyperref}

 
\begin{document}

\begin{figure}
\includegraphics[width=3.7cm]{logolemansU.png}
\hspace{160pt}
\includegraphics[width=3.7cm]{logo_IC2.png}
\end{figure}

\title{\textbf{\textcolor{blue}{Le Mans Université}}\\Licence Informatique 2ème année\\Module 174UP02 Rapport de Projet\\\textbf{LittleRogueNight}}
\author{Maelig Pesantez, Clément Lelandais, Enzo Desfaudais}
\maketitle

\newpage
\tableofcontents

\newpage

\section{Introduction 1-2 pages Maelig}
Ce rapport offre une analyse détaillée du projet LittleRogueNight, un \textbf{rogue-like}\footnote{Rogue-like: genre de jeu vidéo qui s'inspire du jeu original 
"Rogue", caractérisé par des environnements générés de façon procédurale, une difficulté élevée et la mort permanente du personnage du joueur.}, 
jeu vidéo s'inspirant du célèbre jeu Little Nightmares. 
Ce projet est réalisé dans le cadre du module projet, en seconde année de licence informatique au mans.
Réalisé par Clément Lelandais, Enzo Desfaudais et Maelig Pesantez, le jeu est codé du 19/01/2024 au 25/04/2024, soit un peu plus de trois mois.
LittleRogueNight est codé en majorité en langage C, car étudié depuis la première année de licence. Certains langages autre que le langage C sont toutefois présents.
C'est la cas du shell linux, du markdown, du LaTeX et d'autres langages présent en minorité, donc nous verrons plus tard l'utilité.
LittleRogueNight propose une aventure où le joueur incarne un personnage emblématique naviguant à travers des labyrinthes générés aléatoirement, 
peuplés de pièges mortels, de monstres et de boss redoutables. 
Ce jeu offre une expérience unique à chaque partie, offrant ainsi une rejouabilité infinie.\\

LittleRogueNight s'inspire des rogue-like traditionnel, il est donc basé sur les fondamentaux d'un rogue-like classique: labyrinthe parfait
\footnote{Un labyrinthe parfait généré procéduralement est un type de labyrinthe créé de manière aléatoire qui garantit qu'il n'y a qu'un 
seul chemin reliant l'entrée à la fin du labyrinthe.} généré procéduralement\footnote{Un labyrinthe parfait généré procéduralement est un type de labyrinthe 
créé de manière aléatoire qui garantit qu'il n'y a qu'un seul chemin reliant l'entrée à la fin du labyrinthe}, \textit{die and retry}\footnote{Le die and retry est un mécanisme de jeu vidéo, un élément de gameplay, 
qui contraint à jouer puis à perdre pour connaître les mouvements, les actions ou les choix à effectuer pour pouvoir gagner, terminer une partie ou 
un niveau.} et \textit{perma-death}\footnote{De l'anglais permanent death, il signifie que mourrir fait recommencer le joueur depuis le début}. 
Ainsi, le jeu se compose des trois niveaux où chaque labyrinthe est différent à chaque nouvelle partie, 
et où la difficulté augmente au fur et à mesure, et si le personnage meurt à n'importe quel niveau, il recommence au niveau 1.\\

Les mécanismes de déplacement offrent une variété d'actions essentielles telles que: les \textit{dash}\footnote{Un dash représente dans LittleRogueNight un saut continu.} et 
les \textit{drifts}\footnote{Un drift représente dans LittleRogueNight une glissage continue, servant également d'attaque.} qui
ajoutent une dimension tactique aux affrontements. Les graphismes, réalisés en pixel art, apportent une touche rétro à l'esthétique du jeu, 
et renforçent son ambiance immersive.\\

Les combats sont au cœur de l'expérience de jeu, avec un système de barre de vie symbolisée par quatre coeurs qui représentent chacuns vingt-cinq pourcent 
de la vie du joueur. pour les entités antagonistes. Les attaques, 
qu'elles soient directes ou facilitées par des \textit{items}\footnote{Un item, dans le domaine du jeu vidéo, est un objet qui peut être 
collecté par un joueur ou par un personnage non-joueur.} 
spéciaux, proposent des confrontations. Chaque \textit{boss}\footnote{Un boss, dans le domaine du jeu vidéo, est un monstre plus compliqué à battre, 
et se trouvant souvent à la fin d'un niveau} et chaque monstre 
possèdent leur propre barre de vie distincte.\\

Enfin, les graphismes et les mécanismes de jeu s'harmonisent pour créer une expérience, faisant de LittleRogueNight un projet 
combinant éléments classiques du genre rogue-like avec des inspirations qui enrichissent l'expérience de chaque \textit{gameplay}
\footnote{Le gameplay regroupe les caractéristiques d'un jeu vidéo que sont l'intrigue et la façon dont on y joue, par opposition aux effets visuels et sonores. }.\\

\section{Organisation}   
   \subsection{Outils utilisés 1 pages - clement}
   \subsection{Méthodologies et rôles 1 pages - tous}

\section{Conception 1 pages - tous}
   \subsection{Analyse et cahier des charges}
   \subsection{Définition des algorithmes 2 pages - tous}

\section{Développement}
   \subsection{Fonctionnemeent des algorithmes 2 pages - clement}
   \subsection{Structures et mécanismes 1 pages - maelig}
   \subsection{Structures :}

   En langage C, diverses structures sont utilisées dans les fichiers du jeu, notamment les files et les piles. 
   Ces structures jouent un rôle important en matière de stockage des données. Les piles et les files, couramment employées en programmation, 
   présentent des caractéristiques distinctes avec des avantages et des inconvénients spécifiques. Tant les piles que les files possèdent une tête 
   (représentant le premier élément de la liste) ainsi qu'une queue (le dernier élément de la liste).

   Les piles sont des structures dans lesquelles on peut uniquement ajouter des éléments par la tête et retirer des éléments également par la tête 
   (opérations d'empilement et de dépilement). En revanche, les files permettent l'ajout d'éléments par la tête et leur retrait par la queue 
   (opérations d'enfilement et de déflement).

   L'utilisation de piles et de files en langage C facilite la gestion efficace des matrices, simplifiant ainsi la création et la manipulation du 
   labyrinthe dans le jeu. La structure \textit{entity} prend en charge les différentes entités présentes dans le jeu telles que le personnage contrôlé 
   par le joueur, les monstres, ainsi que les objets comme le briquet obtenu à la fin du premier niveau.


   \textbf{Mécanismes :}\\
   \subsection{Outils utilisés 2 pages maelig et enzo}
   \subsection{Fichiers 1 pages enzo}
   
\section{Conclusion 1 a 2 pages Clément et maelig}

\section{Glossaire ou lexique}

\section{Annexe}
   \subsection{Bibliographie}

\end{document}
